
\mainsection{Changes}
This section documents changes made since the ``last'' release of
the software.

\vspace{5mm}

\noindent
This is currently {\bf version 1.10} of the SCALE and MAMBA software.
There are two main components to the system, a run time system called
SCALE,
\begin{center}
  Secure Computation Algorithms from LEuven
\end{center}
and a Python-like programming language called MAMBA
\begin{center}
  Multiparty AlgorithMs Basic Argot
\end{center}
Note, that MAMBA is a type of snake (like a Python), a snake
has scales, and an argot is a ``secret language''.

\vspace{5mm}

\noindent
The software is a major modification of the earlier SPDZ software.
You {\em should not} assume it works the same so please read
the documentation fully before proceeding.

\subsection{Changes in version 1.10 from 1.9}
Apart from the usual minor bug fixes...
\begin{enumerate}
\item There has been some optimization of the low level math routines. This means
	there is a new compile time flag called \verb+MAX_GFP_SIZE+, which needs
	to be set. This corresponds to the maximum size of the finite field used
	for the secret sharing scheme. The variable  \verb+MAX_MOD_SIZE+ corresponds
	to the maximum size of the moduli in the FHE scheme used for the Offline
	phase in the Full Threshold implementation.
\item  A compile time flag \verb+BENCH_OFFLINE+ now prints at the end of the program
	how much offline data your program consumed. By using this data and tweaking
	the settings in \verb|config.h|, and using the \verb|min| and \verb|max| flags
	you can get better latency for your programs (sometimes).
\item   Some extension to the API for the stacks has been introduced, you can
	now peek and poke in relative to the top as well as relative to the bottom
	of the stack.
\item A negative value of verbose larger than $-1$ now prints timing information
	for each instruction in the virtual machine which is executed in the online
	phase.
\item Tidy up of the compiler re location of functions etc. 
	Fixed a lot of documentation, and the definitions in the process. 
	Some functions are now removed/renamed; but these are
	more in the internal functions and less in the exposed functions.
	If a function you have used seems to have changed (i.e. it is not found
	by the compiler) let us know and we will tell you what it has been
        renamed to.
\item You can now convert back and forward between \verb|sint| and \verb|sbit| values
	directly, without having to go via a costly \verb|sregint| intermediate
	value.
\item You can now also convert between an sregint holding an IEEE floating
	point value and an sfloat (and vice-versa).
\item Modifications to how we use the MMO hash function in garbling and in OT
	extension in the main SCALE program.
\item Added instructions for regint bitwise and shift operations, this may speed up some
        programs.
\item Corrected a silly mistake in the random number generation. This gives
	about a 10 percent improvement in throughput for triples generation
	for full threshold access structures; plus minor improvements in
	other cases.
\end{enumerate}

\subsection{Changes in version 1.9 from 1.8}
Apart from the usual minor bug fixes...
\begin{enumerate}
\item The code for interactive BGV key generation for Full Threshold
        setup described in \cite{SPDZKG} has been included.
      See Chapter \ref{sec:keygen} for how this should be used.
\item Offline for Shamir should now be around 25 percent faster.
This improvement also affects other access structures but the effect
is less.
\item The \verb+-D+ compile flag is now turned off by default.
For production code you should enable it.
\item Some optimization to the compile pipeline has been done.
Mainly related to turning off \verb+-M+ by default,
and then crashing the compiler if this looks like
it will cause a problem.
\item Bit decomposition methods which require no statistical
security `gap' have now been implemented.
\item Extended Array's and Matrices for \verb+regint+ and \verb+sregint+ types.\
\item Calling \verb+RETURN()+ from within a tape (and not withing a MAMBA function block) results in a jump to the end of the tape.
\end{enumerate}

\subsection{Changes in version 1.8 from 1.7}
Apart from the usual minor bug fixes...
\begin{enumerate}
\item Added a constructor for \verb|sbit| objects.
\item The ability to execute IEEE compliant floating point arithmetic
      using the Garbled Circuit engine has been added. 
      See Section \ref{sec:ieee}.
\item Loads of bug fixes for SIMD operation on vectors have
      been implemented. Plus a speed up for the SIMD input and
      output of private gfp values.
\item Added a flag \verb|-dOT| to turn off the OT threads to \verb|Player.x|.
      This can increase speed, reduce memory, but means all the GC
      based operations will no longer work.
\item Fixed some bugs in the low level documentation.
\end{enumerate}

\subsection{Changes in version 1.7 from 1.6}
Apart from the usual minor bug fixes...
\begin{enumerate}
  \item There is a {\bf major} change in that we now have two
  compilation paths for MAMBA programs. You can go through the
  old pipe-line, or the new pipeline.
  \begin{itemize}
     \item The old \verb+compile.py+ program has now been renamed
           to \verb+compile-mamba.py+. This is {\em exactly}
           the same as before. It takes a MAMBA program, produces
           the assembly, optimizes the assembly and outputs the
           byte-codes.
           If you only want to use this with default optimizing
           switches, then you can also now use the command
           \verb+compile-old.sh Programs/Name+.
           However, we {\em know} that some of the new functionality
           in relation to stacks etc does not compile correctly with
           the old compiler pipeline, so you have been warned!
     \item A new pipeline uses the old MAMBA compiler to produce the
           assembler, and then the assembler is passed through
           a special scale-assembler called \verb+scasm+.
           The \verb+scasm+ assembler has been built with a collaboration
           with the company Cosmian.
           This assembler is a little more robust than the previous
           version, and has a fully documented optimization step.
           To use this variant with default optimization switches
           use the command \verb+compile-new.sh Programs/Name+
     \item You can make \verb+compile.sh+, which is now the
           default compiler name, point to either the new or old
           compilation path. By default it points to the new
           compilation path.
  \end{itemize}
  A side effect of these changes is that you now also need to
  install Rust, as \verb+scasm+ is written in Rust.
  Full details of the old and new compiler/assembler usage
  are provided in Section \ref{sec:compiler}.
  \item 
  The new assembler can take much longer to compiler stuff, 
  we aim to improve this in the coming months. 
  If this is a problem for you switch to the old pipeline, or
  turn the optimization level of the assembler down a notch
  (see Section \ref{sec:newcompiler} on how to do this). 
  Also if you find bugs {\em please} let us know. 
  \item Almost all the following changes have been done in order
  to accommodate the new assembler, in particular there has been
  a few minor tweaks to the assembler syntax.
  A full description of the byte-codes are now
  given in Section \ref{subsec:instructions}. This table of 
  byte-codes is auto-generated
  from the \verb+scasm+ code, and is thus guaranteed to be up to date with the implementation.
  \item Each virtual `processor' now contains a stack for each
  of the five register types. One can access the stack pointer,
  and manipulate this stack as if it were memory.
  \item The memory management, in particular management of the
  size of memory is now taken out of the MAMBA compiler and
  placed in the runtime. This is safer, but gives a tiny
  performance penalty. In particular if you access (read or write)
  any memory location which is currently `out of range' the
  memory will be automatically resized to ensure no runtime
  error.
  \item The GC operation now works by pushing and popping
  arguments and return values from the stack. See the demo
  program as to how this is done, or the updated documentation
  in Section \ref{sec:GC}.
  \item The Local Function operations work in the same way, via
  the stacks. Again see the demo program, or the updated
  documentation in Section \ref{sec:Local}.
  \item MAMBA functions for opening and closing channels are slightly
  changed.
  \item daBit generation has been improved for all access structures
  for primes $p>2^{64}$ using an adaption of the method in \cite{SPDZKG}.
  This gives both a significant simplification in the code and
  a performance boost.
  \item Added a method to convert \verb|sregint|'s to \verb|sint|'s
  where we treat the \verb|sregint| as an {\em unsigned} 64-bit value.
  \item Added circuits for the SHA-256 and SHA-512 compression functions.
  \item Added more efficient circuits for basic \verb|sregint| arithmetic.
  These operations should now be twice as fast.
 \end{enumerate}

\subsection{Changes in version 1.6 from 1.5}
Apart from the usual minor bug fixes...
\begin{enumerate}
\item Call/Return byte-codes have been added, removing the need for the \verb+JMPI+
instruction. This makes things much simplier going forward we hope.
\item Some byte-code names have been changed, and made more consistent. This
is purely cosmetic, unless you look at the assembler output from the compiler.
\item \verb|sfloat|, \verb|sfix| are now a little more robust. A program will
now definitely abort if the desired security bounds are not met.
We also removed a limitation on the mantissa size in \verb|sfloat|.
\item Fake offline mode is now depreciated and you {\em cannot} select it
from the Setup menu. This is due to a bug, and the fact we never actually
use/test it ourselves.
\item Re the change in version 1.4 for Full Threshold input production.
Turns out the method in SPDZ-2 and in Overdrive are both insecure.
One needs to execute a ZKPoK to prove that party $P_i$'s input is correct;
which is basically the TopGear proof done without a summation. Then
one operates as in SPDZ-2. This has now been alterred.
\item Compiler now outputs assembler for all tapes which are compiled, if
directed to.
\item Direct access to the daBits via a DABIT opcode.
\item Upgrade to the daBit production for dishonest majority.
\item Change to the threading to place sacrificing into the production
threads. This makes the configuration in \verb+config.h+ a little
simpler to understand, and also removes a potential security hole
we discussed in earlier documentation.
This means some of the data structures previously defined in
the RunTime are no longer needed.
\item The Discrete Gaussians can now be selected using different
bounds on the NewHope loop. This bound is defined in \verb|config.h|.
The previous value was hardwired to 20, we now allow the user
to compile the system with any value. The default is now one.
See Section \ref{sec:fhe} for details of this.
\item We have extended the range of ring dimensions available,
so bigger parameters can be utilized. Note, this is untested
in terms of memory usage, so could result in huge memory
and/or network problems.
\item You can also, by editing \verb|config.h| use discrete Gaussian
secret keys instead of short Hamming weight keys if you so desire.
This makes the FHE parameters a little bigger though.
\item We fixed some bugs in the \verb|sfloat| class in relation to
some division operations.
\item We have added a program provided by Mark Will which allows
people using YoSys synthesis tools to produce circuits in Bristol
Fashion.
\end{enumerate}


\subsection{Changes in version 1.5 from 1.4.1}
Apart from the usual minor bug fixes...
\begin{enumerate}
\item New byte-code instruction \verb+SINTBIT+ (and associated function
to utilize it from MAMBA) which sets a given bit of an \verb|sregint|
to a specific \verb|sbit| value.
\item Ability to define and call your own garbled circuit based operations,
for user defined circuits.
See Section \ref{sec:GC} for more details.
\item Ability to define and call your own complex local functions.
See Section \ref{sec:Local} for more details.
\item Extra thread for \verb|aAND| production to make garbled circuit
operations a bit more smooth.
\item Added documentation about SIMD operations.
\item Moved some sections around in the documentation to make it
a little more logical.
\end{enumerate}


\subsection{Changes in version 1.4.1 from 1.4}
Apart from the usual minor bug fixes...
\begin{enumerate}
\item Major bug fix for full threshold, this did not work in 1.4.
\item Comparison of \verb|sregint| values is now done via comparison
	to zero only. Thus overflow errors can occur, but on the other
	hand for general computation this can be much faster.
        This means there are a few byte-code changes from v1.4.
\item Conversion of \verb|sint| to \verb|sregint| for large primes $p$
	on the \verb|sint| side now consume less daBits. This gives
	a bit of a performance boost.
\item We can now convert between \verb|sint| and \verb|sregint|
      when $\log_2 p <64$ as well.
\end{enumerate}

\subsection{Changes in version 1.4 from 1.3}
Apart from the usual minor bug fixes...
\begin{enumerate}
\item We now have a full OT based n-party garbled circuit functionality
integrated. The base Random OT's can either be
derived from the Dual Mode Encryption method of \cite{PVW08}
or the SimpleOT method of \cite{CO15}
The former has, however, a simple attack against it when used to
generate random OT's. Thus we by default utilize the
second method to perform base OTs. This can be changed in
the file \verb+config.h+ if needs be.
\item The base random OTs are  converted into a large number of
correlated COT's, for which we use \cite{AC:FKOS15}[Full Version, Figure 19]
and \cite{C:KelOrsSch15}[Full Version, Figure 7].
These correlated OTs are then converted into random sharings of authenticated
bits (so called aShares/aBits), for this step we use \cite{AC:HazSchSor17}[Full Version, Figure 16].
Finally these aBits are converted into aANDs using
\cite{CCS:WanRanKat17b}[Full Version, Figures 16, 8 and 18 in order].
With the final GC protocol being implemented using \cite{AC:HazSchSor17}.
The hash function in the protocol to generated HaANDs from \cite{CCS:WanRanKat17b}
is implemented using the CCR-secure MMO construction from \cite{GKWY19}.
\item We have extended the MAMBA language to include new datatypes
\verb+sregint+ and \verb+sbit+. These are secure (signed)
64-bit and 1-bit integer datatypes respectively.
These are operated on by using a fixed
set of garbled circuits. As such only certain operations are supported,
roughly equivalent to the kind of arithmetic operations you can
do on C \verb+long+ values. Obviously we could extend the system to
allow user defined circuits to be executed on arbitrary data widths,
but this would detract from our goal of trying to abstract away
much of the nitty-gritty of building MPC solutions.
\item To deal with conversion between LSSS and GC representations
we use the daBit method from \cite{daBitPaper}.
To produce the conversion circuit needed we added a new part to the
Setup routine.
This conversion only works if certain conditions are met by the
prime used for the LSSS scheme; we will discuss this later
in the relevant part of Section \ref{sec:mod2n}.
\item We give advanced users the option of selecting their own
prime for the full-threshold case.
\item Full threshold IO pre-processed data no longer does ZKPoKs,
just as in the original SPDZ-2 implementation. This is secure,
as a moments thought will reveal.
\item The TopGear implementation has been upgraded to use the second
version of the protocol, also other low-level implementation optimizations
have been performed which gives a massive boost to throughput.
\end{enumerate}

\subsection{Changes in version 1.3 from 1.2}
\begin{enumerate}
\item New offline phase for full threshold called TopGear has been
implemented \cite{TopGear}. This means we have changed a number
of FHE related security parameters. In particular the new parameter
sets are {\em more} secure.
{\bf But} they are likely to be different from the ones you have
been using. Thus you will need to generate again FHE parameters
(using \verb+Setup.x+).
\item Bug fixed when using Shamir with a large number of parties
and very very low threshold value.
\item Renaming of some byte-codes to make their meaning clearer.
\item Floating point operations are now (almost) fully supported.
We have basic operations on floating point \verb+sfloat+ variables
supported, as well as basic trigonometric functions. Still to
be completed are the square root, logarithm and exponential functions.
Note that, the implementation of floating point numbers is
different from that in the original SPDZ compiler. The main alteration
is an additional variable to signal error conditions within the
number, and secure processing of this signal.
\end{enumerate}


\subsection{Changes in version 1.2 from 1.1}

\begin{enumerate}
\item A lot of internal re-writing to make things easier to maintain.
\item There are more configuration options in \verb+config.h+ to enable;
      e.g. you can now give different values to the different places
      we use statistical security parameters.
\item Minor correction to how FHE parameters are created. This means
      that the FHE parameters are a bit bigger than they were before.
      Probably a good idea to re-run setup to generate new keys etc
      in the Full Threshold case.
\item Minor change to how \verb+CRASH+ works. If the IO handler returns,
      then the \verb+RESTART+ instruction is automatically called.
\item There is a new run time switch \verb+maxI+ for use when performing
      timings etc. This is only for use when combined with the \verb+max+
      switch below.
\item Two new instructions \verb+CLEAR_MEMORY+ and \verb+CLEAR_REGISTERS+
      have been added.
      These are called from MAMBA via \verb+clear_memory()+ and \verb+clear_registers()+.
      The second {\em may} issue out of order (consider it experimental at present).
\item Bug fix for \verb+sfix+ version of \verb+arcsin/arccos+ functions.
\item When running Shamir with a large number of parties we now move to using
      an interactive method to produce the PRSS as opposed to the non-interactive
      method which could have exponential complexity. This means we can now cope
      with larger numbers of parties in the Shamir sharing case. An example
      ten party example is included to test these routines.
\end{enumerate}


\subsection{Changes in version 1.1 from 1.0}

\begin{enumerate}
\item Major bug fix to IO processing of private input and output.
This has resulted in a change to the byte-codes for these instructions.
\item We now support multiple FHE Factory threads, the precise number
is controlled from a run-time switch.
\item The restart methodology now allows the use to programmatically
create new schedules and program tapes if so desired. The
``demo'' functionality is however exactly as it was before.
Please see the example functionality in the file
\verb+src/Input_Output/Input_Output_Simple.cpp+
\item We have added in some extra verbose output to enable timing of the
offline phase. To time the offline phase, on say bit production,
you can now use the program \verb+Program/do_nothing.mpc+, and then
execute the command for player zero.
\begin{verbatim}
   ./Player.x -verbose 1 -max 1,1,10000000  0 Programs/do_nothing/
\end{verbatim}
Note square production on its own is deliberately throttled
so that when run in a real execution bit production is preferred
over squares.
By altering the constant in the program  \verb+Program/do_nothing.mpc+
you can also alter the number of threads used for this timing operation.
If you enter a negative number for verbose then verbose output
is given for the online phase; i.e. it prints the byte-codes being
executed.
\item The fixed point square root functions have now been extended to
cope with full precision fixed point numbers.
\item The \verb+PRINTxxx+ byte-codes now pass their output via the
\verb+Input_Output+ functionality. These byte-codes are meant for
debugging purposes, and hence catching them via the IO functionality
makes most sense. The relevant function in the IO class
is \verb+debug_output+.
\item We have added the ability to now also input and output \verb+regint+
values via the IO functionality, with associated additional byte-codes
added to enable this.
\item The IO class now allows one to control the opening and closing of
channels, this is aided by two new byte-codes for this purpose
called \verb+OPEN_CHANNEL+ and \verb+CLOSE_CHANNEL+.
\item Input of clear values via the IO functionality (i.e.
for \verb+cint+ and \verb+regint+ values) is now internally checked to ensure that
all players enter the same clear values.
Note, this requires modification to any user defined
derived classes from \verb+Input_Output_Base+.
See the chapter on IO for more details on this.
\item The way the chosen IO functionality is bound with the main program
has now also been altered. See the chapter on IO for more details on this.
\item These changes have meant there are a number of changes to the
specific byte-codes, so you will need to recompile MAMBA programs.
If you generate your own byte-codes then your backend will need to
change as well.
\end{enumerate}


\subsection{Changes From SPDZ}
Apart from the way the system is configured and run there are
a number of functionality changes which we outline below.

\subsubsection{Things no longer supported}
\begin{enumerate}
\item We do not support any $GF(2^n)$ arithmetic in the run time
environment. The compiler will no longer compile your programs.
\item There are much fewer switches in the main program, as we want
to produce a system which is easier to support and more useful in
building applications.
\item Socket connections, file, and other forms of IO to the main MPC
engine is now unified into a single location. This allows {\em you} to
extend the functionality without altering the compiler or
run-time in any way (bar changing which IO class you load
at compile time). See Section \ref{sec:IO} for details.
\end{enumerate}

\subsubsection{Additions}
\begin{enumerate}
\item The offline and online phases are now fully integrated.
This means that run-times will be slower than you would have got
with SPDZ, but the run-times obtained are closer to what you would
expect in a ``real'' system.
{\bf Both} the online and offline phases are {\bf actively} secure
with abort.
\item Due to this change it can be slow to start a new instance
and run a new program. So we provide a new (experimental) operation which
``restarts'' the run-time. This is described in Section \ref{sec:restart}.
This operation is likely to be revamped and improved in the next
release as we get more feedback on its usage.
\item We support various Q2 access structures now, which can be
defined in various ways: Shamir threshold, via Replicated sharing,
or via a general Monotone Span Programme (MSP).
For replicated sharing you can define the structure
via either the minimally qualified sets, or the maximally unqualified
sets.
For general Q2-MSPs you can input a non-multiplicative MSP and the
system will find an equivalent multiplicative one for you using the
method of \cite{CDM00}.
\item Offline generation for Q2 is done via Maurer's method \cite{Maurer}, but
for Replicated you can choose between Maurer and the reduced communication
method of Keller, Rotaru, Smart and Wood \cite{KRSW}.
For general Q2-MSPs, and Shamir sharing, the online phase is the method described in Smart and Wood \cite{SW18},
with ({\em currently}) the offline phase utilizing Maurer's multiplication method \cite{Maurer}.
\item All player connections are now via SSL, this is not strictly
needed for full threshold but is needed for the other access structures
we now support.
\item We now have implemented more higher level mathematical functions for
the \verb+sfix+ datatype, and corrected a number of bugs.
A similar upgrade is expected in the next release for the \verb+sfloat+ type.
\end{enumerate}

\subsubsection{Changes}
\begin{enumerate}
\item The {\bf major} change is that the offline and online phases are now integrated.
This means that to run quick test runs, using full threshold is going to take
ages to set up the offline data. Thus for test runs of programs in the online
phase it is best to test using one of the many forms of Q2 access structures.
For example by using Shamir with three players and threshold one. Then once your
online program is tested you can move to a production system with two players
and full threshold if desired.
\item You now compile a program by executing
\begin{verbatim}
      ./compile.py Programs/tutorial
\end{verbatim}
where \verb+Programs/tutorial+ is a {\em directory} which contains
a file called \verb+tutorial.mpc+. Then the compiler puts all of the
compiled tapes etc {\em into this directory}.
This produces a much cleaner directory output etc.
By typing \verb+make pclean+ you can clean up all pre-compiled directorys
into their initial state.
\item The compiler picks up the prime being used for secret sharing
after running the second part of \verb+Setup.x+. So you need to recompile
the \verb+.mpc+ files if  you change the prime used in secret sharing, and you
should not compile any SCALE \verb+.mpc+ programs before running \verb+Setup.x+.
\item Internally (i.e. in the C++ code), a lot has been re-organized. The major simplification
is removal of the \verb+octetstream+ class, and it's replacement by a combination
of \verb+stringstream+ and \verb+string+ instead. This makes readibility
much easier.
\item All opcodes in the range \verb+0xB*+ have been changed, so any byte-codes
you have generated from outside the python system will need to be changed.
\item We have tried to reduce dependencies between classes in the
C++ code a lot more. Thus making the code easier to manage.
\item Security-wise we use the latest FHE security estimates for the
FHE-based part, and this can be easily updated. See Chapter \ref{sec:fhe} on FHE
security later.
\end{enumerate}
