\mainsection{Installation and Setup}

\subsection{Installation}\label{subsec:install}
You will require the following previously loaded programs and
libraries:
\begin{itemize}
\item GCC/G++: Tested with version 8.3.1
\item MPIR library, compiled with C++ support (use flag --enable-cxx when running configure) : Tested with version 3.0.0
\item Python, ideally with `gmpy2` package (for testing): Tested with Python 2.7.17
\item CPU supporting AES-NI and PCLMUL
\item OpenSSL: Tested with version 1.1.0.b
\item Crypto++: Tested with version 7.0
\item yasm (for MPIR)
\item m4 (for MPIR)
\item A Rust compiler of at least version 1.39.0 (4560ea788 2019-11-04)
\end{itemize}
Developers will also require
\begin{itemize}
\item clang-format as to apply the standard C++ format to files. Tested with clang-format version 6.0.0.
\end{itemize}

\subsubsection{Installing and running using nix-shell}
To make things easy we have a quick build system via \verb|nix-shell| which may help you
if you do not want to install all the above yourself.
First you need to install \verb|nix-shell| if you don't already have it (may need a re-login to update env vars).
\begin{verbatim}
    curl -L https://nixos.org/nix/install | sh
\end{verbatim}
Then invoke \verb|nix-shell| to get a fully ready development environment with all libraries installed.
This will automatically download all the dependencies and tools you need.

Due to this setup, you also don't need any custom configuration in \verb+CONFIG.mine+, instead
you just copy \verb+CONFIG+ to \verb+CONFIG.mine+ and remove the \verb+OSSL+ variable entirely and 
replace the \verb+ROOT = something+ with \verb+ROOT = ..+ (yes the two dots are on purpose).

Inside that shell, you can compile a program in the \verb+Programs+ directory by invoking
\begin{verbatim}
    ./compile.sh Programs/test_fix_array
\end{verbatim}
You can now jump to Section \ref{sec:compilesh} (although you might want to read \ref{sec:CONFIG}
as well for other compilation tweaks).

\subsubsection{Installing Rustc}

Go to \url{https//rustup.rs} to find the best installation command for your platform.
For linux this may be...
\begin{verbatim}
    curl --proto '=https' --tlsv1.2 -sSf https://sh.rustup.rs | sh
\end{verbatim}
For our Ubuntu systems we had to use the binary installer marked
\verb+x86_64-unknown-linux-gnu+ from the page
\url{https://forge.rust-lang.org/infra/other-installation-methods.html}
as there was some incompatibility between \verb+curl+ and our
\verb+openssl+ installation.

\subsubsection{Installing MPIR, OpenSSL and Crypto++}
This bit, on explaining how to install  MPIR, OpenSSL and Crypto++ inside \verb+$HOME/local+,
is inspired from \href{https://rdragos.github.io/2019/01/07/scale/}{this} blogpost.
The target directory here can be changed to whatever you wish.
If you follow this section we assume that you have \textbf{cloned}
the main repository in your \verb+$HOME+ directory.
\begin{verbatim}
    mylocal="$HOME/local"
    mkdir -p ${mylocal}
    cd ${mylocal}

    # install MPIR 3.0.0
    curl -O 'http://mpir.org/mpir-3.0.0.tar.bz2'
    tar xf mpir-3.0.0.tar.bz2
    cd mpir-3.0.0
    ./configure --enable-cxx --prefix="${mylocal}/mpir"
    make && make check && make install

    # install OpenSSL 1.1.0
    cd $mylocal
    curl -O https://www.openssl.org/source/openssl-1.1.0j.tar.gz
    tar -xf openssl-1.1.0j.tar.gz
    cd openssl-1.1.0j
    ./config --prefix="${mylocal}/openssl"
    make && make install

    # install crypto++
    curl -O https://www.cryptopp.com/cryptopp820.zip
    unzip cryptopp820.zip -d cryptopp820
    cd cryptopp820
    make && make install PREFIX=${mylocal}/cryptopp
\end{verbatim}
Now export MPIR, OpenSSL and Crypto++ paths by copying the following
lines at the end of your \verb+$HOME/.bashrc+ configuration file.
\begin{verbatim}
    # this goes at the end of your $HOME/.bashrc file
    export mylocal="$HOME/local"

    # export OpenSSL paths
    export PATH="${mylocal}/openssl/bin/:${PATH}"
    export C_INCLUDE_PATH="${mylocal}/openssl/include/:${C_INCLUDE_PATH}"
    export CPLUS_INCLUDE_PATH="${mylocal}/openssl/include/:${CPLUS_INCLUDE_PATH}"
    export LIBRARY_PATH="${mylocal}/openssl/lib/:${LIBRARY_PATH}"
    export LD_LIBRARY_PATH="${mylocal}/openssl/lib/:${LD_LIBRARY_PATH}"

    # export MPIR paths
    export PATH="${mylocal}/mpir/bin/:${PATH}"
    export C_INCLUDE_PATH="${mylocal}/mpir/include/:${C_INCLUDE_PATH}"
    export CPLUS_INCLUDE_PATH="${mylocal}/mpir/include/:${CPLUS_INCLUDE_PATH}"
    export LIBRARY_PATH="${mylocal}/mpir/lib/:${LIBRARY_PATH}"
    export LD_LIBRARY_PATH="${mylocal}/mpir/lib/:${LD_LIBRARY_PATH}"

    # export Crypto++ paths
    export CPLUS_INCLUDE_PATH="${mylocal}/cryptopp/include/:${CPLUS_INCLUDE_PATH}"
    export LIBRARY_PATH="${mylocal}/cryptopp/lib/:${LIBRARY_PATH}"
    export LD_LIBRARY_PATH="${mylocal}/cryptopp/lib/:${LD_LIBRARY_PATH}"
\end{verbatim}

\subsubsection{Change CONFIG.mine}
\label{sec:CONFIG}
We now need to copy the file \verb+CONFIG+ in the main directory to the file
\verb+CONFIG.mine+.
Then we need to edit \verb+CONFIG.mine+, so as to place the correct
location of this ROOT directory correctly,
as well as indicating where the OpenSSL library should be picked up
from (this is likely to be different from the
system installed one which GCC would automatically pick up).
This is done by executing the following commands
\begin{verbatim}
    cd $HOME/SCALE-MAMBA
    cp CONFIG CONFIG.mine
    echo "ROOT = $HOME/SCALE-MAMBA" >> CONFIG.mine
    echo "OSSL = ${mylocal}/openssl" >> CONFIG.mine
\end{verbatim}
You can also at this stage specify various compile time options
such as various debug and optimisation options.
We would recommend commenting out all DEBUG options from FLAGS
and keeping \verb+OPT = -O3+.
\begin{itemize}
\item The \verb+DEBUG+ flag is a flag which turns on checking for
reading before writing on registers, thus it is mainly a flag
for development testing of issues related to the compiler.
\item The \verb+DETERMINISTIC+ flag turns off the use of true randomness.
This is really for debugging to ensure we can replicate errors due.
It should {\bf not} be used in a real system for obvious reasons.
\end{itemize}
If you are going to use full threshold LSSSs then \verb+MAX_MOD+
needs to be set large enough to deal with the sizes of the FHE
keys. Otherwise this can be set to just above the word size of your
secret-sharing modulus to obtain better performance.
As default we have set it for use with full threshold.
The value \verb+MAX_GFP+ corresponds to the size of the
prime used for the secret sharing scheme (in $64$-bit words).

\subsubsection{Change compile.sh}
\label{sec:compilesh}
You may want to edit \verb+compile.sh+ to change from
the default new compilation pipeline to the old one.

\subsubsection{Change config.h}
If wanted you can also now configure various bits of the system
by editing the file
\begin{verbatim}
     config.h
\end{verbatim}
in the sub-directory \verb+src+.
The main things to watch out for here are the various FHE security parameters;
these are explained in more detail in Section \ref{sec:fhe}.
Note, to configure the statistical security parameter for the number representations
in the compiler (integer comparison, fixed point etc) from the default
of $40$ you need to add the following commands to your MAMBA programs.
\begin{verbatim}
    program.security = 100
    sfix.kappa = 60
    sfloat.kappa = 30
\end{verbatim}
However, in the case of the last two you {\em may} also need to change the
precision or prime size you are using. See the documentation for
\verb+sfix+ and \verb+sfloat+ for this.

\subsubsection{Final Compilation}
The only thing you now have to do is type
\begin{verbatim}
    make progs
\end{verbatim}
That's it! After make finishes then you should see a |PlayerBinary.x|
executable inside the SCALE-MAMBA directory.

\subsection{Creating and Installing Certificates}
\label{sec:certs}
For a proper configuration you need to worry about the rest
of this section.
However, for a quick idiotic test installation jump down to the ``Idiot
Installation'' of Section \ref{sec:idiot}.

All channels will be TLS encrypted. For SPDZ this is not needed, but for
other protocols we either need authenticated or secure channels. So might
as well do everything over {\em mutually} authenticated TLS. We are going
to setup a small PKI to do this. You thus first need to create
keys and certificates for the main CA and the various players you
will be using.

When running \verb+openssl req ...+ to create certificates, it is
vitally important to ensure that each player has a different Common
Name (CN), and that the CNs contain no spaces.  The CN is used later
to configure the main MPC system and be sure about each party's
identity (in other words, they really are who they say they are).

~~

\noindent
First go into the certificate store
\begin{verbatim}
       cd Cert-Store
\end{verbatim}
Create CA authority private key
\begin{verbatim}
       openssl genrsa -out RootCA.key 4096
\end{verbatim}
Create the CA self-signed certificate:
\begin{verbatim}
       openssl req -new -x509 -days 1826 -key RootCA.key -out RootCA.crt
\end{verbatim}
Note, setting the DN for the CA is not important, you can leave them
at the default values.

~~

\noindent
Now for {\em each} MPC player create a player certificate, e.g.
\begin{verbatim}
       openssl genrsa -out Player0.key 2048
       openssl req -new -key Player0.key -out Player0.csr
       openssl x509 -req -days 1000 -in Player0.csr -CA RootCA.crt \
              -CAkey RootCA.key -set_serial 0101 -out Player0.crt -sha256
\end{verbatim}
remembering to set a different Common Name for each player.

In the above we assumed a global shared file system.  Obviously on
a real system the private keys is kept only in the
\verb+Cert-Store+ of that particular player, and the player public
keys are placed in the \verb+Cert-Store+ on each player's
computer. The global shared file system here is simply for test
purposes. Thus a directory listing of \verb+Cert-Store+
for player one, in a four player installation, will look like
\begin{verbatim}
      Player1.crt
      Player1.key
      Player2.crt
      Player3.crt
      Player4.crt
      RootCA.crt
\end{verbatim}


\subsection{Running Setup}\label{subsec:setup}
The program \verb+Setup.x+ is used to run a one-time setup
for the networking and/or secret-sharing system being used
and/or set up the GC to LSSS conversion circuit.
You must do networking before secret-sharing (unless you keep
the number of players fixed), since the secret-sharing setup
picks up the total number of players you configured when setting
up networking.
And you must do secret sharing setup before creating the conversion
circuit (since this requires the prime created for the secret
sharing scheme).
\begin{itemize}
\item Just as above for OpenSSL key-generation, for demo purposes we assume
a global file store with a single directory \verb+Data+.
\end{itemize}
Running the program \verb+Setup.x+ and specifying the secret-sharing
method will cause the program to generate files holding MAC and/or FHE
keys and place them in the folder \verb+Data+.  When running the
protocol on separate machines, you must then install the appropriate
generated MAC key file \verb+MKey-*.key+ in the \verb+Data+ folder of
each player's computer.  If you have selected full-threshold, you also
need to install the file \verb+FHE-Key-*.key+ in the same directory.
You also need to make sure the public data files
\verb+NetworkData.txt+ and \verb+SharingData.txt+ are in the directory
\verb+Data+ on each player's computer.
These last two files specify the configuration which you select with
the \verb+Setup.x+ program.

~~

\noindent We now provide more detail on each of the three aspects of the program
\verb+Setup.x+.

\subsubsection{Data for networking}
Input provided by the user generates the file
\verb+Data/NetworkData.txt+ which defines the following
\begin{itemize}
\item The root certificate name.
\item The number of players.
\item For each player you then need to define
\begin{itemize}
  \item Which IP address is going to be used
  \item The name of the certificate for that player
\end{itemize}
\iffalse XXXX
\item Whether a fake offline phase is going to be used.
\item Whether a fake sacrifice phase is going to be used.
\fi
\end{itemize}

\subsubsection{Data for secret sharing:}
You first define whether you are going to be using full threshold (as in
traditional SPDZ), Shamir (with $t<n/2$), a Q2-Replicated scheme, or
a Q2-MSP.

\paragraph{Full Threshold:}
In this case the prime modulus cannot be chosen directly, but
needs to be selected to be FHE-friendly\footnote{In all other cases you select the prime modulus for the LSSS directly at this point.}.
Hence, in this case we give you two options.
\begin{itemize}
\item Recommended: You specify the number of bits in the modulus
(between 16 bits and 1024 bits).  The system will then
search for a modulus which is compatible with the FHE system we are
using.
\item Advanced: You enter a specific prime. The system then searches
for FHE parameters which are compatible, if it finds none (which is highly
likely unless you are very careful in your selection) it aborts.
\end{itemize}
After this stage the MAC keys and FHE secret keys are setup and written into the
files
\begin{center}
\verb+MKey-*.key+ and \verb+FHE-Key-*.key+
\end{center}
in the \verb+Data+ directory.
This is clearly an insecure way of parties picking their MAC keys. But this is only a
research system.
At this stage we also generate a set of keys
for distributed decryption of a level-one FHE scheme if needed.

\textbf{NOTE:} Section \ref{sec:keygen} describes how to perform a secure distributed 
FHE key generation to avoid relying on a trusted third party as the above does.

\iffalse XXXX
For the case of fake offline we assume these keys are on {\em each} computer,
but using fake offline is only for test purposes in any case.
\fi

\paragraph{Shamir Secret Sharing:}
Shamir secret sharing we assume is self-explanatory.
For the Shamir setting we use an online phase using the reduced communication
protocols of \cite{KRSW};
the offline phase ({\em currently}) only supports {\em Maurer}'s multiplication method
\cite{Maurer}.
This will be changed in future releases to also support the new offline method from
\cite{SW18}.


\paragraph{Replicated Secret Sharing:}
For Replicated sharing you should enter a complete monotone Q2 access structure.
There are three options to do this,
\begin{enumerate}
\item As a set of maximally unqualified sets;
\item As a set of minimally qualified sets;
\item As a simple threshold system.
\end{enumerate}
If the first (resp. second) option is selected, then any set that is neither
a superset nor a subset of a set provided as input will be assumed qualified (resp. unqualified).
The last option is really for testing, as most
threshold systems implemented using Replicated secret-sharing will be
less efficient than using Shamir.
Specifying either the first or second option and providing input
results in the program computing all of the qualified and unqualified
sets in the system.

For Replicated secret-sharing you
can also decide what type of offline phase you want: one based on
{\em Maurer}'s multiplication method \cite{Maurer},
or one based on our {\em Reduced} communication protocols \cite{KRSW}.

~~

\noindent
Suppose we want to input the four-party Q2 access structure given by
\[\Gamma^- = \{\{P_1, P_2\}, \{P_1, P_3\}, \{P_1, P_4\}, \{P_2, P_3, P_4\}\}\]
and
\[\Delta^+ = \{\{P_1\}, \{P_2, P_3\}, \{P_2, P_4\}, \{P_3, P_4\}\}.\]
We can express this example in the following two ways:
\begin{verbatim}
4 parties, maximally unqualified
        1 0 0 0
        0 1 1 0
        0 1 0 1
        0 0 1 1
\end{verbatim}
or as
\begin{verbatim}
4 parties, minimally qualified
        1 1 0 0
        1 0 1 0
        1 0 0 1
        0 1 1 1
\end{verbatim}

\noindent
As a second example with six parties with a more complex
access structure for our Reduced Communication protocol consider:
\[\Gamma^- = \{\{P_1, P_6\}, \{P_2, P_5\}, \{P_3, P_4\}, \{P_1, P_2, P_3\}, \{P_1,
  P_4, P_5\}, \{P_2, P_4, P_6\}, \{P_3, P_5, P_6\}\}\]
and
\[\Delta^+ = \{\{P_1, P_2, P_4\}, \{P_1, P_3, P_5\}, \{P_2, P_3, P_6\}, \{P_4, P_5, P_6\}\}.\]
Each party is in a different pair of sets.  We can represent it via:
\begin{verbatim}
6 parties, maximally unqualified sets
        1 1 0 1 0 0
        1 0 1 0 1 0
        0 1 1 0 0 1
        0 0 0 1 1 1
\end{verbatim}

\paragraph{Q2-MSP Programs:}
A final way of entering a Q2 access structure is via a MSP, or equivalently,
via the matrix which defines an underlying Q2 LSSS.
For $n$ parties we define the number of shares each party has to be
$n_i$, and the dimension of the MSP to be $k$. The MSP is then given
by a $(\sum n_i) \times k$ matrix $G$.
The first $n_1$ rows of $G$ are associated with player one, the
next $n_2$ rows with player two and so on.
An secret sharing of a value $s$ is given by the vector of values
\[ \vs = G \cdot \vk \]
where $\vk = (k_i) \in \F_p^k$ and $\sum k_i =s$.
A subset of parties $A$ is qualified if the span of the rows they control
contain the vector $(1,\ldots,1) \in \F_p^k$.

This secret sharing scheme can be associated with a monotone
access structure, and we call this scheme Q2 if the associated
access structure is Q2.
However, it is not the case (unlike for Shamir and Replicated sharing)
that the Q2 MSP is itself multiplicative (which is crucial for
our MPC protocols).
Thus if you enter a Q2 MSP which is {\bf not} multiplicative, we
will automatically extend this for you into an equivalent multiplicative MSP
using the method of \cite{CDM00}.

As in the Shamir setting we use an online phase using the reduced communication
protocols of \cite{KRSW};
the offline phase ({\em currently}) only supports {\em Maurer}'s multiplication method
\cite{Maurer}.
This will be changed in future releases to also support the new offline method from
\cite{SW18}.

In the file \verb+Auto-Test-Data/README.txt+ we provide some examples
of MSPs. If we take the MSP for Shamir (say) over three parties with threshold
one we obtain the matrix
\[
   G = \left( \begin{array}{cc}
      1 & 2 \\ 1 & 3 \\ 1 & 4
       \end{array} \right).
\]
This is slightly different from the usual Shamir generating matrix
as we assume the target vector is $(1,1)$ and not $(1,0)$ as is often
the case in Shamir sharing.


\subsubsection{Conversion Circuit}
Here, as most users will not have a VHDL compiler etc, we provide
a generic 512 bit conversion circuit for turning an LSSS sharing
into a GC sharing.
This needs to be specialised to which ever prime $p$ has been selected above.
To do this we use some C-code  which specializes the generic 512
bit circuit for this specific prime chosen, and then tries
to simplify the resulting circuit. This simplification is rather
naive, and can take a long time, but usually results in a circuit ten percent
less in size than the original (in terms of the all important AND gates).
One can modify how long is spent in the simplification by
changing a variable in \verb+config.h+.

\subsection{Idiot's Installation}
\label{sec:idiot}
To install an installation of the three party Shamir based
variant, with default certificates and all parties running
on local host.
Execute the following commands
\begin{verbatim}
     cp Auto-Test-Data/Cert-Store/* Cert-Store/
     cp Auto-Test-Data/1/* Data/
\end{verbatim}
You can play with writing your own MAMBA programs and
running them on the same local host (via IP address
127.0.0.1).
