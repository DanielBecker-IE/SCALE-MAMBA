\mainsection{Standard Library}


\subsection{Boxes}
A box is essentially an array of size one. It is a `safe' way for users
to write/read from the SCALE memory one element at a time.
To load the standard library version of a Box use
\begin{lstlisting}
     use scale_std::heap::Box;
\end{lstlisting}
As it is tied to the SCALE memory it only comes in the
types  $\ints$, $\SecretI$, $\ClearModp$ and $\SecretModp$.

\func{Box::uninitialized()}
To initialise a box with no pre-defined value use
\begin{lstlisting}
    let mut a= Box<SecretI64>::uninitialized();
\end{lstlisting}
\warning{This will consume one storage locations in the user memory.}

\func{a.set(x)}
Does the obvious.

\func{a.get(i)}
Again does the obvious.


\subsection{Arrays}
This data type is implemented using the SCALE memory and not
via the vectorized instructions.
It is thus only available for $\ints$, $\SecretI$, $\ClearModp$
and $\SecretModp$ data types.
You can always use simple arrays of the form
\begin{lstlisting}
    let key128: [SecretI64; 2] = [zero, none];
\end{lstlisting}
as in the earlier AES example. But these will not compile as
soon as the size becomes too large. The simple arrays will
compile to registers, and thus be more efficient than arrays
stored in memory. But arrays stores in memory are more flexible.
To load the standard library version of Arrays use
\begin{lstlisting}
     use scale_std::array::Array;
\end{lstlisting}

\func{Array::uninitialized()}
To initialise an array with no pre-defined values use
\begin{lstlisting}
    let mut a= Array<SecretI64, 25>::uninitialized();
\end{lstlisting}
\warning{This will consume $25$ storage locations in the user memory.}


\func{a.fill(x)}
Fills an array, used with initialization
\begin{lstlisting}
    let a: Array<SecretI64, 25> = Array::fill(zero);
\end{lstlisting}


\func{a.set(i,x)}
Does the obvious, but has no bound checks, i.e. you can access the 50th element of a 25 element array.

\func{a.get(i)}
Again does the obvious, but also has no bound checks.

\func{a.iter()}
Iterator over an \verb|Array|,
\begin{lstlisting}
    for (i, val) in a.iter().enumerate() {
        ....
    }
\end{lstlisting}
To iterate backwards use
\begin{lstlisting}
    for (i, val) in a.iter().rev().enumerate() {
        ....
    }
\end{lstlisting}

\func{a.reveal()}
For the secret types this creates a non-secret version, as in.
\begin{lstlisting}
     let cint_array = sint_array.reveal();
\end{lstlisting}

\func{a.slice(range)}
Converts (part of) an array to a slice, of which the length is only known at runtime.
\begin{lstlisting}
    let a_slice: Slice<SecretI64> = a.slice(3..);
\end{lstlisting}

\subsubsection{Stack Operations on Arrays}
One can also apply the stack operations on an array, which
applies to all the elements in the array.
Note the pop operation works in reverse, so that pushing and
then popping gives the same array.
The stack read operations (pop, peek etc) will pop and peek
the entire array, so you need to ensure the stack really has
that many elements on it; i.e. no access is performed outside
the stack, otherwise a runtime crash will occur.

\func{push(x), pop()}
\func{peek(i), poke(i, x)}
\func{peek_from_top(i), poke_from_top(i, x)}
These are called using the following convention
\begin{lstlisting}
    unsafe { Array::push(&a) }
    let a = unsafe { Array::<SecretI64, 25>::pop() };
\end{lstlisting}

\subsection{Slices}
\label{sec:slices}
This data type is equal to Arrays, the only difference being that the length of a slice is not known at compile time.
Just like Arrays, this data type is implemented using the SCALE memory and not
via the vectorized instructions.
As such it is only available for $\ints$, $\SecretI$, $\ClearModp$
and $\SecretModp$ data types.
To load the standard library version of slices use
\begin{lstlisting}
     use scale_std::slice::Slice;
\end{lstlisting}
\warning{In general using slices will consume more user memory than Arrays.}

\func{Slice::uninitialized()}
To initialise a slice with no pre-defined values use
\begin{lstlisting}
    let mut a= Slice<SecretI64>::uninitialized();
\end{lstlisting}

\func{s.set(i,x)}
Does the obvious.

\func{s.get(i)}
Again does the obvious

\func{s.iter()}
Iterator over an \verb|Slice|,
\begin{lstlisting}
    for (i, val) in s.iter().enumerate() {
        ....
    }
\end{lstlisting}
To iterate backwards use
\begin{lstlisting}
    for (i, val) in a.iter().rev().enumerate() {
        ....
    }
\end{lstlisting}

\func{s.reveal()}
For the secret types this creates a non-secret version, as in.
\begin{lstlisting}
     let cint_slice = sint_slice.reveal();
\end{lstlisting}

\func{s.slice(range)}
Returns part of the original slice.
\begin{lstlisting}
    let s_slice: Slice<SecretI64> = s.slice(3..);
\end{lstlisting}

\func{s.len()}
Returns the length of the slice as a u64-number.
\begin{lstlisting}
    let length: u64 = s.len();
\end{lstlisting}

\subsection{Matrices}
\label{sec:matrices}
Again this data type is implemented using the SCALE memory and not
via the vectorized instructions.
It is thus only available for $\ints$, $\SecretI$, $\ClearModp$
and $\SecretModp$ data types.
To load the standard library version of Matrices use
\begin{lstlisting}
     use scale_std::matrix::Matrix;
\end{lstlisting}
There are a similar set of functions as for the Array type.

\func{Matrix::unitialized()}
To initialise a matrix with no pre-defined values, with
fives rows and four columns, use
\begin{lstlisting}
    let mut a= Matrix<SecretI64, 5, 4>::unitialized();
\end{lstlisting}


\func{a.fill(x)}
Fills an array, used with initialization
\begin{lstlisting}
    let a: Matrix<SecretI64, 5, 4> = Matrix::fill(zero);
\end{lstlisting}


\func{a.set(i,j,x)}
Does the obvious.

\func{a.get(i,j)}
Again does the obvious

\func{a.get_row(i)}
Does the obvious, returning the result as an \verb|Array|.

\func{a.get_column(i)}
Again does the obvious


\func{a.iter()}
Iterator over a \verb|Matrix|,
\begin{lstlisting}
    for (i, val) in a.iter().enumerate() {
        ....
    }
\end{lstlisting}
The iterator works in the order
$a_{0,0}, a_{0,1}, a_{0,2}, \ldots, a_{0,c-1}, a_{1,0}, a_{1,1}, \ldots, a_{r-1,c-1}$.

\noindent
To iterate backwards use
\begin{lstlisting}
    for (i, val) in a.iter().rev().enumerate() {
        ....
    }
\end{lstlisting}

\func{a.reveal()}
For the secret types this creates a non-secret version, as in.
\begin{lstlisting}
     let cint_matrix = sint_matrix.reveal();
\end{lstlisting}

\subsubsection{Stack Operations on Matrices}
One can also apply the stack operations on a matrix, which
applies to all the elements in the matrix.
The push operation works in the same order as the above
iterator, with the pop operations working in reverse.
The stack read operations (pop, peek etc) will pop and peek
the entire matrix, so you need to ensure the stack really has
that many elements on it; i.e. no access is performed outside
the stack, otherwise a runtime crash will occur.

\func{push(x), pop()}
\func{peek(i), poke(i, x)}
\func{peek_from_top(i), poke_from_top(i, x)}
These are called using the following convention
\begin{lstlisting}
    unsafe { Matrix::push(&a) }
    let a = unsafe { Matrix::<SecretI64, 5, 4>::pop() };
\end{lstlisting}

\subsection{ClearIEEE}
This class enables one to load and print IEEE floating point values.
It is mainly a helper class for the \verb|SecretIEEE| class below.
The usage is demonstrated in the example below:
\begin{lstlisting}
    use scale_std::secret_ieee::*;
    #[inline(always)]
    fn main() {
      let c= ClearIEEE::from(3.141592);
      print!(c,"\n");
    }
\end{lstlisting}

\subsection{SecretIEEE}
This class gives direct access to the IEEE-754 compliant implementation
of floating point numbers in SCALE. Recall these are implemented using
evaluations of binary circuits, so are not as fast as the \verb|sfloat|
and \verb|sfix| types implemented in MAMBA (which are not currently available
in the Rust interface, but we hope to fix this soon).

\func{SecretIEEE::from(x)}
You can convert a $\SecretI$ or a \verb|ClearIEEE| to a \verb|SecretIEEE|
as follows:
\begin{lstlisting}
    let c= ClearIEEE::from(3.141592);
    let sc=SecretIEEE::from(c);

    let i: i64= 1677216;
    let si= SecretI64::from(i);
    let sfi=SecretIEEE::from(si);
\end{lstlisting}
You can also convert a  \verb|SecretIEEE| to a $\SecretI$, which rounds the
floating point value to a 64-bit integer (if possible).
\begin{lstlisting}
     let ic=SecretI64::from(sc);
\end{lstlisting}

\func{s.reveal()}
For the \verb|SecretIIEEE| type this creates a \verb|ClearIEEE| version, as in.
\begin{lstlisting}
     let ci = si.reveal();
\end{lstlisting}


\subsubsection{Arithmetic}
The following operations are allowed between two elements of type  \verb|SecretIEEE|:
\begin{verbatim}
     +   -   *   /
\end{verbatim}

\subsubsection{Comparisons}
The following `operators' can be applied between two \verb|SecretIEEE| values
The output is a $\SecretBit$ value.
As the result of the operator cannot be used in a conditional branch,
we use the member function notation for such `operators', as opposed
to the operator notation. Thus syntactically the programmer is less
likely to make a mistake.
\func{a.gt(x), a.ge(x), a.lt(x), a.le(x), a.gt(x), a.eq(x), a.ne(x)}

\noindent
The following give variants which compare just to zero.
\func{a.gtz(), a.gez(), a.ltz(), a.lez(), a.gtz(), a.eqz(), a.nez(x)}


\subsection{ClearInteger}
We define $\Zk$ as the set of integers $\{x \in \Z: -2^{k-1} \le x \le 2^{k-1}-1\}$,
which we embed into $\F_p$ via the map $x \mapsto x \pmod{p}$.
In MAMBA the user had to manually think of elements in $\F_p$
as representing values in $\Zk$, and manually keep track of their
sizes for all the `advanced' integer operations.
This was efficient, but prone to errors and misunderstandings.
In Rust we have a seperate type to represent elements in $\Zk$,
both in clear and secret form.
This is strongly typed in the sense that a
\verb|ClearInteger<3>|  is a different type from a
\verb|ClearInteger<4>|.

This class enables one to load and print clear integers, and acts
mainly as a helper class for the \verb|SecretInteger| class below.
The usage is demonstrated in the example below:
\begin{lstlisting}
    use scale_std::integer::*;
    #[inline(always)]
    fn main() {
      let c: ClearInteger<3>= ClearInteger::from(2 as i64);
      print!(c,"\n");
    }
\end{lstlisting}

\func{ClearInteger::from(x)}
You can convert a $\cint$ or a $\ints$ to a \verb|ClearInteger<K>| as follows:
\begin{lstlisting}
    let c = ClearModp::from(2);
    let ci: ClearInteger<3>= ClearInteger::from(c);
    let ci1: ClearInteger<3>= ClearInteger::from(2 as i64);
\end{lstlisting}

\func{a.recast()}
A value of one size can be recast to another using the \verb|recast|
member function. It is assumed the user knows what they are doing here
and that such recasting will not result in an invalid representation
being created.
\begin{lstlisting}
    let c = ClearModp::from(2);
    let ci: ClearInteger<3>= ClearInteger::from(c);
    let ci1: ClearInteger<5>= a.recast();
\end{lstlisting}

\func{a.rep()}
Returns the internal representation as a \verb|ClearModp| value.

\func{a.Mod2m(M,S)}
This computes the value of $a$ modulo $2^M$.
When $S=\true$ the value $a$ is assumed to lie in $\Zk$,
but when $S=\false$ the value $a$ is assumed to lie in $[0,\ldots,2^{K-1})$.
Mathematically this computes the expression
\[  \left( a + S \cdot 2^{K-1} \right) \pmod 2^M. \]
Implicitly this assumes that $M < K$.
\begin{lstlisting}
    let a: ClearInteger<10> = ClearInteger::from(-24);
    let b = a.Mod2m(ConstU64::<5>, ConstBool::<true>);
\end{lstlisting}

\func{a.Mod2(S)}
As above but takes $M=1$.



\subsection{SecretInteger}
Many operations on \verb|SecretInteger|s make use of the statistical
security gap \verb|KAPPA| defined in the pre-amble.

\func{SecretInteger::from(x)}
You can convert a $\cint$ or a \verb|ClearInteger<K>| to a
\verb|SecretInteger<K>|
as follows:
\begin{lstlisting}
    let c = ClearModp::from(2);
    let ci: ClearInteger<3>= ClearInteger::from(c);
    let si1: SecretInteger<3>= SecretInteger::from(c);
    let si2: SecretInteger<3>= SecretInteger::from(ci);
\end{lstlisting}

\func{a.recast()}
A value of one size can be recast to another using the \verb|recast|
member function. It is assumed the user knows what they are doing here
and that such recasting will not result in an invalid representation
being created.
\begin{lstlisting}
    let c = ClearModp::from(2);
    let si: SecretInteger<3>= SecretInteger::from(c);
    let si1: SecretInteger<5>= a.recast();
\end{lstlisting}

\func{a.rep()}
Returns the internal representation as a \verb|SecretModp| value.

\func{a.reveal()}
For the \verb|SecretInteger| typs this creates a \verb|ClearInteger| version, as in.
\begin{lstlisting}
     let ci = si.reveal();
\end{lstlisting}

\func{a.Mod2m(M,S)}
This computes the value of $a$ modulo $2^M$.
When $S=\true$ the value $a$ is assumed to lie in $\Zk$,
but when $S=\false$ the value $a$ is assumed to lie in $[0,\ldots,2^{K-1})$.
Mathematically this computes the expression
\[  \left( a + S \cdot 2^{K-1} \right) \pmod 2^M. \]
Implicitly this assumes that $M < K$.
\begin{lstlisting}
    let a: SecretInteger<10> = SecretInteger::from(-24);
    let b = b.Mod2m(ConstU64::<5>, ConstBool::<true>);
\end{lstlisting}

\func{a.Mod2(S)}
As above but takes $M=1$.

\func{Pow2::<K,KAPPA>(b)}
If $b$ is a $\sint$ in the range $[0,\ldots,K)$ this computes
the value $2^b$ as a \verb|SecretInteger<K>|
\begin{lstlisting}
    let b = SecretModp::from(28);
    let d = Pow2::<32,40>(b);
\end{lstlisting}

\func{a.TruncPr(M,S)}
Implements the probabilistic trunction operation of rounding $a/2^M$.
\begin{lstlisting}
    let a = SecretModp::from(23);
    let b: SecretInteger::<6> = SecretInteger::from(a);
    let c = b.TruncPr(ConstU64::<3>, ConstBool::<true>);
\end{lstlisting}

\func{a.Trunc(M,S)}
Implements the deterministic trunction operation of rounding $a/2^M$.
\begin{lstlisting}
    let a = SecretModp::from(23);
    let b: SecretInteger::<6> = SecretInteger::from(a);
    let c = b.Trunc(ConstU64::<3>, ConstBool::<true>);
\end{lstlisting}

\func{a.TruncRoundNearest(M,S)}
Implements the deterministic rounding to the nearest integer operation
of rounding $\lceil a/2^M \rfloor$.
\begin{lstlisting}
    let a = SecretModp::from(23);
    let b: SecretInteger::<6> = SecretInteger::from(a);
    let c = b.TruncRoundNearest(ConstU64::<3>, ConstBool::<true>);
\end{lstlisting}

\func{a.ObliviousTrunc(m)}
Implements \verb|Trunc| operation where $m$ is a $\sint$ value
which lies in the range $[0,\ldots,K)$.
\begin{lstlisting}
    let a = SecretModp::from(23);
    let b: SecretInteger::<6> = SecretInteger::from(a);
    let m = SecretModp::from(3);
    let c = b.ObliviousTrunc(m);
\end{lstlisting}

\subsubsection{Comparisons}
The following `operators' can be applied between two \verb|SecretInteger<K>| 
values,the output being a $\SecretBit$ value.
As the result of the operator cannot be used in a conditional branch,
we use the member function notation for such `operators', as opposed
to the operator notation. Thus syntactically the programmer is less
likely to make a mistake.
\func{a.gt(x), a.ge(x), a.lt(x), a.le(x), a.gt(x), a.eq(x), a.ne(x)}

\noindent
The following give variants which compare just to zero.
\func{a.gtz(), a.gez(), a.ltz(), a.lez(), a.gtz(), a.eqz(), a.nez(x)}

\subsection{Arithmetic of ClearInteger and SecretInteger}
Since a reduction operation (to ensure the value really lies in
$\Zk$) is expensive we give function and operator versions of arithmetic.
\warning{In the member function versions no reduction occurs, thus 
these can be used when the user knows that no wrap around will occur.
We thus mark them as {\bf unsafe}}.
The operator versions allow for ease of use, and guarantee correctness,
but will be slower than the function versions.

The operators unarray minus (i.e. negation), addition, subtraction
and multiplication are supported. If a clear and secret value are
combined then the result is secret.
The equivalent function versions for arithmetic have the following signatures:
\begin{lstlisting}
  ClearInteger<K>::negate()                          -> ClearInteger<K>;
  ClearInteger<K>::add(ClearInteger<K>)              -> ClearInteger<K>;
  ClearInteger<K>::sub(ClearInteger<K>)              -> ClearInteger<K>;
  ClearInteger<K>::mul(ClearInteger<K>)              -> ClearInteger<K>;
  ClearInteger<K>::add_secret(SecretInteger<K>)      -> SecretInteger<K>;
  ClearInteger<K>::sub_secret(SecretClearInteger<K>) -> SecretInteger<K>;
  ClearInteger<K>::mul_secret(SecretInteger<K>)      -> SecretInteger<K>;
  SecretInteger<K>::negate()                         -> SecretInteger<K>;
  SecretInteger<K>::add(SecretInteger<K>)            -> SecretInteger<K>;
  SecretInteger<K>::sub(SecretInteger<K>)            -> SecretInteger<K>;
  SecretInteger<K>::mul(SecretInteger<K>)            -> SecretInteger<K>;
  SecretInteger<K>::add_clear(ClearInteger<K>)       -> SecretInteger<K>;
  SecretInteger<K>::sub_clear(ClearInteger<K>)       -> SecretInteger<K>;
  SecretInteger<K>::mul_clear(ClearInteger<K>)       -> SecretInteger<K>;
\end{lstlisting}
